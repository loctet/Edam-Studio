
\section{GUI Usage}

The EDAM Studio GUI (available at \url{http://localhost:3000} when running via Docker) 
supports editing EDAM models, generating Solidity code, and verifying model behavior.

\textbf{Editing the model.}
The model can be edited in three ways:
\begin{itemize}
  \item \textbf{Force-Directed Graph} or \textbf{GraphViz}: 
    The model is displayed as a graph. 
    Edit by dragging nodes and edges, or use the sidebar to add transitions and states.
  \item \textbf{Manual Editor}: 
    Enter the EDAM specification as text in the \texttt{Manual Editor} tab, 
    then click \texttt{View EDAM} to load it into the editor. 
    The text format uses a compact syntax (states, roles, variables, and transitions).
    If a model is saved in a .edam file, it can be loaded by clicking \texttt{Load EDAM File} 
    and selecting the file, conversaly, the model can be saved by clicking \texttt{Save EDAM File} 
    and selecting the file.
    The following is the format of a .edam file:
    \begin{lstlisting}
ModelName
Role1,Role2,Role3
variable1:type1, variable2:type2
[from_state] {userVar:role:mode}, guard, [external_calls] \
callerVar:operation(params){assignments} {user:role:mode} [to_state]
    \end{lstlisting}
    First line is the model name, second line is the roles, third line is the variables, 
    and the following lines are the transitions.
    The transitions are formatted as follows:
    \begin{lstlisting}
[from_state] {userVar:role:mode}, guard, [external_calls] \
callerVar:operation(params){assignments} {user:role:mode} [to_state]
    \end{lstlisting}
    The \texttt{from\_state} is the source state, the \texttt{userVar:role:mode} is the user and role and mode, 
    the \texttt{guard} is the guard expression, the \texttt{external\_calls} are the external calls, 
    the \texttt{callerVar:operation(params)} is the caller variable and operation and parameters, 
    the \texttt{assignments} are the assignments, and the \texttt{to\_state} is the target state.

  \item \textbf{JSON Editor}: 
  The bottom panel provides a JSON editor for direct structural editing. 
  Changes are reflected in the graph.
\end{itemize}

\textbf{Generating code.}
Add the desired model(s) to the \texttt{Generation List} via \texttt{Add to Generation List}, then click \texttt{Generate Code} in the header. After generation, the results appear with two tabs:
\begin{itemize}
\item \textbf{Generated Code}: View the generated Solidity contracts. Use \texttt{Download Generated ZIP} to save the code (contracts, tests, and configuration) locally.
\item \textbf{Run Solidity Test}: Execute the generated tests on the server and view pass/fail results.
\end{itemize}

\textbf{Running a trace test.}
To verify model behavior without generating full code, click \texttt{Run A Trace Test} (bottom-right). Enter trace commands in the text area (e.g., \texttt{p1>Token1.deploy [] [10]}) and click \texttt{Test Trace} to execute. The tool simulates the trace against the model and reports the outcome.


\section{Reproduction of Experimental Claims}

For each experimental claim in the paper, 
the following commands and procedures enable reproduction:

\subsection{Code Generation}

\textbf{Claim}: EDAM Studio generates Solidity smart contracts from EDAM models.

\noindent\textbf{Reproduction}: Using the Docker container:
\begin{lstlisting}
docker exec -it edam-studio edam-cli \
  assettransfer c20 basicprovenance --mode 2
\end{lstlisting}
Generated code (a ZIP file named \texttt{<ModelName>\_<new participant probability>\_<id>.zip} containing the generated code, tests, and configuration files) appears in \texttt{/app/Studio/Generated-code/} in the Docker container. 
For custom trace counts (as in the paper):
\begin{lstlisting}
docker exec -it edam-studio edam-cli c20 amm --mode 3 \
  --number_symbolic_traces 1000 --number_transition_per_trace 40
\end{lstlisting}

\noindent\textbf{Command template}:
\begin{lstlisting}
docker exec -it edam-studio edam-cli <model1> [model2] ... \
  --mode <1|2|3|4> [options]
\end{lstlisting}

Table~\ref{tab:cli-params} lists all optional parameters for code generation.
\begin{table}[ht]
\centering
\caption{Full parameters for \texttt{edam-cli} code generation.}
\label{tab:cli-params}
\begin{tabular}{@{}llp{6.5cm}@{}}
\hline
\textbf{Parameter} & \textbf{Default} & \textbf{Description} \\
\hline
\multicolumn{3}{l}{\textit{Required}} \\
\texttt{--mode} & --- & Generation mode: 1 (single trace), 2 (few traces), 3 (many traces), 4 (all models) \\
\hline
\multicolumn{3}{l}{\textit{Symbolic trace generation}} \\
\texttt{--number\_symbolic\_traces} & 200 & Number of symbolic traces to generate per model \\
\texttt{--number\_transition\_per\_trace} & 10 & Maximum transitions per symbolic trace \\
\texttt{--number\_real\_traces} & 5 & Number of real traces for validation \\
\texttt{--max\_fail\_try} & 2 & Maximum retries on trace generation failure \\
\hline
\multicolumn{3}{l}{\textit{Probability parameters}} \\
\texttt{--probability\_new\_participant} & 0.35 & Probability of adding a new participant in a trace \\
\texttt{--probability\_right\_participant} & 0.7 & Probability of selecting the correct participant \\
\texttt{--probability\_true\_for\_bool} & 0.5 & Probability of \texttt{true} for boolean values \\
\hline
\multicolumn{3}{l}{\textit{Value bounds for generated data}} \\
\texttt{--min\_int\_value} & 0 & Minimum integer value in generated tests \\
\texttt{--max\_int\_value} & 100 & Maximum integer value in generated tests \\
\texttt{--max\_gen\_array\_size} & 10 & Maximum size of generated arrays \\
\texttt{--min\_gen\_string\_length} & 5 & Minimum length of generated strings \\
\texttt{--max\_gen\_string\_length} & 10 & Maximum length of generated strings \\
\hline
\multicolumn{3}{l}{\textit{Test generation options}} \\
\texttt{--add\_pi\_to\_test} & off & Add process invariants to generated tests \\
\texttt{--add\_test\_of\_state} & on & Include state assertions in tests \\
\texttt{--add\_test\_of\_variables} & on & Include variable assertions in tests \\
\hline
\texttt{--z3\_check\_enabled} & on & Enable Z3 SMT solver checks during trace generation \\
\hline
\end{tabular}
\end{table}

\subsection{Pre-generated Code and Data}

\textbf{Claim}: The data in \texttt{EXPERIMENT\_DATA/} 
corresponds to the paper's experiments.

\noindent\textbf{Reproduction}: 
The archives \texttt{Generated Code.zip}, \texttt{Mutations.zip}, and \texttt{Excels Mutations Results.zip} contain the exact outputs used in the paper. No regeneration needed for verification. To regenerate:
\begin{enumerate}
  \item Generate code for all models: 
  \texttt{edam-cli assettransfer c20 amm basicprovenance ... --mode 3}
  \item Run ReSuMo on each generated ZIP: 
  \texttt{.run resumo <zip\_filename>} (via \texttt{cli\_commands.py})
  \item Merge results using \texttt{ReSuMo/scripts/merger\_to\_one\_excel\_recap.py}
\end{enumerate}

\subsection{Mutation Testing Results}

\textbf{Claim}: Mutation scores and timing reported in tables/figures.

\noindent\textbf{Reproduction}: Open \texttt{EXPERIMENT\_DATA/Mutation 
result - Recaps.xlsx} for summary data. 
Detailed per-operator results are in \texttt{Excels Mutations Results.zip}. 
To reproduce mutation testing on a single model, 
using the Docker container:

\begin{lstlisting}
docker exec -it edam-studio bash
source venv/bin/activate
export OPAMROOT=/root/.opam && eval $(opam env --root=/root/.opam)
cd /app/Studio
python3 CLI/cli_commands.py run resumo <ModelName>_<version>_<id>.zip
\end{lstlisting}
(Replace \texttt{<ModelName>\_<version>\_<id>.zip} 
with an actual generated ZIP filename from \texttt{Generated-code/}.)

\subsection{Test Execution}

\textbf{Claim}: Generated tests execute successfully and achieve coverage.

\noindent\textbf{Reproduction}:
\begin{lstlisting}
docker exec -it edam-studio python3 CLI/cli_commands.py run test <zip_file> test
\end{lstlisting}
For coverage: \texttt{run test <zip\_file> coverage}.

\subsection{Timing Expectations}

Results may differ from the paper due to:
\begin{itemize}
\item Machine speed (CPU, disk I/O)
\item Non-deterministic symbolic trace generation (Z3 solver, random sampling)
\item Node/npm version differences
\end{itemize}
Code generation: typically 5--20 min per model. 
Testing: up to 20 min for 10k tests. 
Mutation: less than 2 h for most models; AMM can take days.